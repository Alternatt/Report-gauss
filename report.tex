%% -*- coding: utf-8 -*-
\documentclass[12pt,a4paper]{scrartcl} 
\usepackage[T2A]{fontenc}           % кодировка
\usepackage[utf8]{inputenc}         % кодировка исходного текста
\usepackage[english,russian]{babel} % локализация и переносы

\usepackage{color} % подключить пакет color
% выбрать цвета
\definecolor{Black}{RGB}{0,0,0}
\definecolor{Blue}{RGB}{0,0,128}
% назначить цвета при подключении hyperref
\usepackage[unicode, colorlinks, urlcolor=Blue, linkcolor=Black, pagecolor=Black]{hyperref}

\usepackage{indentfirst}
\usepackage{misccorr}
\usepackage{graphicx}
\usepackage{amsmath}

\begin{document}
\begin{titlepage}
		\begin{center}
			\large
			МИНИСТЕРСТВО НАУКИ И ВЫСШЕГО ОБРАЗОВАНИЯ РОССИЙСКОЙ ФЕДЕРАЦИИ
			
			Федеральное государственное бюджетное образовательное учреждение высшего образования
			
			\textbf{АДЫГЕЙСКИЙ ГОСУДАРСТВЕННЫЙ УНИВЕРСИТЕТ}
			\vspace{0.25cm}
			
			Инженерно-физический факультет
			
			Кафедра автоматизированных систем обработки информации и управления
			\vfill

			\vfill
			
			\textsc{Отчет по практике}\\[5mm]
			
			{\LARGE Программаная реализация численного метода \textit{Решение системы линейных алгебраических уравнений методом Гаусса.}}
			\bigskip
			
			1 курс, группа 1ИВТ2
		\end{center}
		\vfill
		
		\newlength{\ML}
		\settowidth{\ML}{«\underline{\hspace{0.7cm}}» \underline{\hspace{2cm}}}
		\hfill\begin{minipage}{0.5\textwidth}
			Выполнил:\\
			\underline{\hspace{\ML}} П.\,Д.~Александрович\\
			«\underline{\hspace{0.7cm}}» \underline{\hspace{2cm}} 2023 г.
		\end{minipage}%
		\bigskip
		
		\hfill\begin{minipage}{0.5\textwidth}
			Руководитель:\\
			\underline{\hspace{\ML}} С.\,В.~Теплоухов\\
			«\underline{\hspace{0.7cm}}» \underline{\hspace{2cm}} 2023 г.
		\end{minipage}%
		\vfill
		
		\begin{center}
			Майкоп, 2023 г.
		\end{center}
  \end{titlepage}



\tableofcontents %Оглавление

\newpage

\section{Введение}
\label{sec:intro}

% Что должно быть во введении
\begin{enumerate}
 \item Текстовая формулировка задачи
 \item Код, решающий данную задачу
 \item Скриншот решения программы
\end{enumerate}
\section{Текстовая формулировка задачи} 
\textbf {Задание:} 

Решение системы линейных алгебраических уравнений методом Гаусса.

\textbf{Теория:}

Метод Гаусса\cite{Метод Гаусса:1}. — классический метод решения системы линейных алгебраических уравнений (СЛАУ). Это метод последовательного исключения переменных, когда с помощью элементарных преобразований система уравнений приводится к равносильной системе треугольного вида, из которой последовательно, начиная с последних (по номеру), находятся все переменные системы.

\textbf{Алгоритм решения СЛАУ методом Гаусса подразделяется на два этапа:}

На первом этапе осуществляется так называемый прямой ход, когда путём элементарных преобразований над строками систему приводят к ступенчатой или треугольной форме, либо устанавливают, что система несовместна. А именно, среди элементов первого столбца матрицы выбирают ненулевой, перемещают его на крайнее верхнее положение перестановкой строк и вычитают получившуюся после перестановки первую строку из остальных строк, домножив её на величину, равную отношению первого элемента каждой из этих строк к первому элементу первой строки, обнуляя тем самым столбец под ним. После того, как указанные преобразования были совершены, первую строку и первый столбец мысленно вычёркивают и продолжают пока не останется матрица нулевого размера. Если на какой-то из итераций среди элементов первого столбца не нашёлся ненулевой, то переходят к следующему столбцу и проделывают аналогичную операцию.

На втором этапе осуществляется так называемый обратный ход, суть которого заключается в том, чтобы выразить все получившиеся базисные переменные через небазисные и построить фундаментальную систему решений, либо, если все переменные являются базисными, то выразить в численном виде единственное решение системы линейных уравнений. Эта процедура начинается с последнего уравнения, из которого выражают соответствующую базисную переменную (а она там всего одна) и подставляют в предыдущие уравнения, и так далее, поднимаясь по «ступенькам» наверх. Каждой строчке соответствует ровно одна базисная переменная, поэтому на каждом шаге, кроме последнего (самого верхнего), ситуация в точности повторяет случай последней строки.

\section{Ход работы}

\subsection{Код, решающий данную задачу} 
\subsubsection{\textbf{Прямой ход метода Гаусса:}}
\begin{enumerate}
    \item 
        Поиск максимального элемента
    \item 
        Обмен строками
    \item 
        Приведение к верхнеугольному виду
    \item 
        Проверка совместности системы
\end{enumerate}
\begin{verbatim}
def gauss(A, B):
    # Прямой ход метода Гаусса
    n = len(B)
    for i in range(n):
        # Поиск максимального элемента в столбце i
        maxEl = abs(A[i][i])
        maxRow = i
        for k in range(i + 1, n):
            if abs(A[k][i]) > maxEl:
                maxEl = abs(A[k][i])
                max_ryad = k
        # Обмен строками
        for k in range(i, n):
            T = A[max_ryad][k]
            A[max_ryad][k] = A[i][k]
            A[i][k] = T
        T = B[max_ryad]
        B[max_ryad] = B[i]
        B[i] = T
        # Приведение к верхнетреугольному виду
        for k in range(i + 1, n):
            c = -A[k][i] / A[i][i]
            for j in range(i, n):
                if i == j:
                    A[k][j] = 0
                else:
                    A[k][j] += c * A[i][j]
            B[k] += c * B[i]
    # Проверка совместности системы
    for i in range(n):
        if A[i][i] == 0 and B[i] != 0:
            return 0 # Система несовместна
\end{verbatim}
\newpage
\subsubsection{\textbf{Обратный ход метода Гаусса:}}
\begin{verbatim}
    # Обратный ход метода Гаусса
    x = [0] * n
    for i in range(n - 1, -1, -1):
        x[i] = B[i]
        for j in range(i + 1, n):
            x[i] -= A[i][j] * x[j]
        x[i] /= A[i][i]
    return x
\end{verbatim}
\subsubsection{\text{Определение матрицы:}}
\textbf{Выбираем или изменяем, для примера (совместная и несовместная):}
\begin{verbatim}
# Определяем матрицу системы уравнений (ЗДЕСЬ НУЖНО ВЫБРАТЬ МАТРИЦУ)

#Пример совместной. --------------------------------
'''
A = [[1, 1, 1], [1, -1, 2], [2, -1, -1]]
'''
#Пример несовместной. ------------------------------
'''
A = [[2, 3, -1], [4, 6, -2], [1, 2, -1]]
'''
# Определяем столбец свободных членов----------------

B = [6, 5, -3]
\end{verbatim}
\subsubsection{\textbf{Вывод результата:}}
\begin{verbatim}
x = gauss(A, B)
if x == 0:
    print("Система уравнений несовместна.")
else:
    print("Результат:")
    print(x)
\end{verbatim}

\newpage

\subsection{Скриншоты решения программы}
\label{sec:picexample}
\begin{figure}[h]
	\centering
	\includegraphics[width=0.71\textwidth]{code.png}
	\caption{Код с совместной матрицей}\label{fig:par}
\end{figure}
\label{sec:picexample}
\begin{figure}[h]
	\centering
	\includegraphics[width=0.71\textwidth]{code2.png}
	\caption{Код с несовместной матрицей}\label{fig:par}
\end{figure}
\newpage
\begin{thebibliography}{9}
\bibitem{Метод Гаусса:1} Метод Гаусса: \href{https://ru.wikipedia.org/wiki/Метод_Гаусса}{https://ru.wikipedia.org/wiki/Метод\_Гаусса}

\end{thebibliography}

\end{document}
